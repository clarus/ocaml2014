\documentclass[hyperref={pdfpagelabels=false}]{beamer}
\usepackage{beamerthemesplit}
\usepackage[utf8]{inputenc}
\usepackage[T1]{fontenc}
\usepackage{lmodern}
\usepackage{graphicx}
\usepackage{amsmath, amsfonts, amssymb}
\usepackage{tikz}

\usetheme{Green}

\begin{document}
  \title{CoqOfOCaml}
  \author{Guillaume Claret}
  \date{5 September 2014}
  \maketitle

  \section{Introduction}
  \begin{frame}
    \frametitle{Two languages}
    \emph{OCaml}:
    \begin{itemize}
      \item programming language
      \item functional programming with imperative features
      \item many libraries and programs
    \end{itemize}
    \emph{Coq}:
    \begin{itemize}
      \item mainly used as a proof language
      \item purely functional programming (not even non-termination)
      \item limited number of libraries for the programmer
    \end{itemize}
  \end{frame}
  \begin{frame}
    \frametitle{Bridges}
    \emph{Coq} to \emph{OCaml}: extraction mechanism:
    \begin{itemize}
      \item by Pierre Letouzey
    \end{itemize}

    \emph{OCaml} to \emph{Coq}:
  \end{frame}
  \begin{frame}
    \frametitle{Introduction}
    \begin{block}{CoqOfOCaml}
      A compiler from \emph{OCaml} to \emph{Coq}.
    \end{block}
  \end{frame}
\end{document}